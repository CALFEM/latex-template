\chapter{Introduction}
Acoustic fatigue, also known as sonic fatigue, is a problem for the aircraft industry. Structural elements, such as skin surface panels, on aircraft may be exposed to high intensity sound levels. These high sound levels may cause the structural element to vibrate, simply because sound is pressure fluctuations. This means that the high sound intensities act as fluctuating force loading on the structural element. The vibration may lead to cracks in the structural element and already existing cracks may grow. If this vibration is allowed to go on unchecked it may eventually lead to failure. The vibrational frequencies involved tend to be in the order of hundreds of Hz. Acoustic fatigue can therefore develop quickly as it does not take a long time before a large number of cycles have passed. 

There are several possible sources for this high intensity load. The perhaps most obvious source is noise from the jet engine or the propeller. But there is also the case where geometrical features cause strong loads. These may be control surfaces, flaps and cavities. All these geometrical features include separated flow and periodic vortex shedding as key features. 

Acoustic fatigue is a problem in several ways. Obviously, aircraft suffering from failure in flight can have drastic consequences. Keeping cracks under control is therefore important. During maintenance the aircraft can be tested for cracks, but not all parts of the aircraft can easily be tested. Frequent maintenance requirements are undesired for economic reasons. 

Traditionally, acoustic fatigue has been dealt with using empirical methods. Design guidelines have been developed and are applied. An example of this is the ESDU design guidelines on acoustic fatigue~\cite{ESDU}. The current design guidelines have three main limitations:
\begin{enumerate} \itemsep1pt \parskip0pt \parsep0pt \vspace{-5pt}
  \item\label{list:intro1} The load levels at the eigenfrequencies of the panel structure must be known.
  \item\label{list:intro2} The load is assumed to be fully in phase over the exposed structure, or in other words, the load is assumed to have a uniform spatial distribution.
  \item\label{list:intro3} The design guidelines are limited to a simple surface panel with linear response. 
\end{enumerate}
\vspace{-5pt}
The first limitation can arguably be said to be outside the scope of the design guideline and not being a limitation to the guidelines per se. However, the load levels are often determined experimentally. Finding the load levels experimentally from the use of wind-tunnels or flight testing is expensive. It is also desired to handle acoustic fatigue early in the design phase before any flight testing can be done. It is much cheaper to deal with an issue early on compared to when the aircraft has gone to flight testing. The second limitation reduces the accuracy of the response prediction of the structure which in turn reduces the accuracy of the fatigue prediction. The response prediction has been shown to improve significantly if a more realistic load distribution is applied~\cite{Campos,Cunnigham_exp2}. The third limitation restricts the guidelines to certain structures. For example, composite panels can have large non-linear response which is not covered by the design guidelines.

For the reasons mentioned above, it is desired to use numerical simulations to predict acoustic fatigue. When it comes to determining the dynamical properties of the aircraft structures, the Finite Element Method (FEM) has been in heavy use for some time now. This avoids the third limitation with design guidelines given above. However, the determination of the load levels and their spatial and temporal characteristics using numerical methods appears to have been less dealt with. This is the domain of Computational Fluid Dynamics (CFD). Most, if not all, of the typical load sources have been studied with CFD. There are many CFD studies on engine noise, cavities etc. Usually, the concern has been on what noise they make or aspects unrelated to acoustic fatigue. The previous of the two is an important application of CFD within the subject of acoustic fatigue. In other words, while having the ability to make accurate fatigue predictions are desirable, removing the source is much better. It can also be used to address limitation number one in the list above. However, attempts to study the intensities \emph{and} the spatial distribution of the  pressure fluctuations on the exposed surfaces directly as a tool for making fatigue predicitions is often not attempted. For the important cases of separated flow, there are few numerical studies on the surface pressure fluctuations. Those that exist are performed on Reynolds numbers far below realistic ones for aircraft operations. Also, they do not make the step to do a response prediction of the exposed surface.

The general aim of this dissertation is to improve the load prediction for acoustic fatigue by attempting to use numerical methods in the form of CFD. % to determine intensity as well as the spatial and temporal characteristics of the load. 
This can then be applied to an FE model of the exposed structure. %The main focus is in this dissertation is on the load prediction using CFD. 
It is desired to capture both the load levels and the spatial distribution of the load as both are important parameters that have good potential to improve the response predictions. Knowledge of what is required of the numerical method to produce an accurate load prediction is naturally sought. {\em In paper A, the simulated load is applied to an FE simulation of the response of a simple structure. This is done to estimate how effective a load obtained from CFD can be in the response prediction. However, the focus in this dissertation is on the load prediction rather than the response prediction. A secondary objective is also to increase the understanding of the flow mechanisms that cause the load.} 

The scope of the dissertation is limited to the loads caused by separated flows. This excludes all loads that appear in the far field. %An example of this is a jet engine exhaust at some distance to the exposed structure. 
%Open cavities are also an example of separated flows wich have the additional phenomenon of Rossiter~\cite{Rossiter} tones. Neither open cavities nor Rossiter tones are considered in this dissertation.

It should also be mentioned, that while the aim of this dissertation is on acoustic fatigue, the work here should be of interest for efforts to reduce the noise inside the cabin as well. The sound pressure levels discussed are high and the vibration of the outer skin surface panels are clearly a transmission path. Although cabin noise reduction is not directly inside the scope of this dissertation, it should be noted that this work may be beneficial in other contexts as well. 

This dissertation is organised in two parts. The second part consists five research papers that have been produced in this project. These papers are preceded by an extended introduction and overview of the work. It also contains a more thorough treatment of topics that are treated in less detail in the papers. Chapter~\ref{sec:acousticFatigue} is a literature review on acoustic fatigue. In Chapter~\ref{sec:fluids}, the motion of fluids is discussed. This includes a phenomenological discussion of turbulent flow, the numerical treatment as well as an introduction to flow types dealt with in this dissertation. The numerical methods used to predict the structural response in paper A is covered in Chapter~\ref{sec:structures}. Chapter~\ref{sec:paperSummary} contains a summary of the appended papers. Finally, some concluding remarks and suggestions for future work are given in Chapter~\ref{sec:conclusions}.


%The term acoustic or sonic fatigue as well as discussing sound levels in relation to the aircraft exterior during flight may be a bit misleading. Sound is nothing but pressure waves. However, sound is often separated from hydrodynamic pressure. When the propagation speed of sound is much higher than the convection speed, the hydrodynamic pressure and sound pressure tends to act fairly independent of each other. This is true in most situations in daily life. However, at the Mach numbers that jet engined aircraft typically operate in this separation is a lot less clear. 




%Here there will be an introduction. Somewhere there will likely be a mentioning of this \cite{Chiu} article to give an example of acoustic fatigue found in the literature. Something else I would like to cite is \cite{Morton}. It is an overview of the methods used for certifying the F-22 aircraft for high cycle and sonic fatigue.

%\section{Aim and Objective}
%Hopefully I can come up with something resembeling some aim or objective.

%\section{Thesis Outline}
%Here I shall explain what the reader can find where.
